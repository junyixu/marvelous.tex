% 勒让德多项式

\section{勒让德多项式}%
\begin{equation}
  \laplacian \varphi = 0
\end{equation}
由 \cref{eq:广义曲线坐标系拉普拉斯算子} 写出求坐标系的拉普拉斯算子,
\begin{equation}
  \laplacian = 
  \frac{1}{r^2} \pdv{r} r^2 \pdv{r} +
  \frac{1}{r^2} \pdv{\theta} \sin \theta \pdv{\theta} +
  \frac{1}{r^2 \sin \theta} \pdv[2]{\phi}
  \label{eq:球坐标系的拉普拉斯算子}
\end{equation}
当函数$\varphi(r, \theta)$与\(\phi\)无关的时候
\begin{equation}
  \begin{aligned}
    \laplacian &= 
  \frac{1}{r^2} \pdv{r} r^2 \pdv{r} +
  \frac{1}{r^2 \sin\theta} \pdv{\theta} \sin \theta \pdv{\theta} +
  \frac{1}{r^2 \sin \theta} \pdv[2]{\phi} \\
\text{(当函数$\varphi(r, \theta)$与\(\phi\)无关的时候)}
\quad &=
  \frac{1}{r^2} \pdv{r} r^2 \pdv{r} +
  \frac{1}{r^2 \sin\theta} \pdv{\theta} \sin \theta \pdv{\theta}
  \end{aligned}
\end{equation}
\begin{equation}
  \begin{aligned}
   & \left(\cancel{\frac{1}{r^2}} \pdv{r} r^2 \pdv{r} +
  \frac{1}{\cancel{r^2} \sin\theta} \pdv{\theta} \sin \theta \pdv{\theta}\right)
  \varphi \\
  &=
  \left(\pdv{r} r^2 \pdv{r} +
  \frac{1}{\sin\theta} \pdv{\theta} \sin \theta \pdv{\theta}\right)
  \varphi = 0
  \end{aligned}
\end{equation}
若 \(\varphi(r, \theta) = R(r) \Theta(\theta)\)
\begin{equation}
 \frac{1}{R} \red{\left(\pdv{r} r^2 \pdv{r} \right) }R=
-\frac{1}{\Theta} \blue{\left(\frac{1}{\sin\theta} \pdv{\theta} \sin \theta \pdv{\theta}\right)} \Theta
  = \lambda
\end{equation}
\begin{align}
  \red{\left(\dv{r} r^2 \dv{r} \right) } R = \lambda R \label{eq:Legendre_R}\\
\blue{\left(\frac{1}{\sin\theta} \dv{\theta} \sin \theta \dv{\theta}\right)}\Theta =  - \lambda \Theta \label{eq:Legendre_Q}
\end{align}
\cref{eq:Legendre_R} 即
\begin{equation}
  \red{\left(\dv{r} r^2 \dv{r} \right) } R = \lambda R 
\end{equation}
的 $r> 0$ 而指数函数\(e^t>0, t\in \RR\),不妨设\(r=e^t\)。
由于
链式法则
\begin{equation}
  \begin{aligned}
  &\dv{f}{r}= \dv{t}{r} \dv{f}{t} = \frac{1}{r} \dv{f}{t} \\
  \implies &\dv{r}=\frac{1}{r} \dv{t}
  \end{aligned}
\end{equation}
得
\begin{equation}
  \red{\left(\dv{r} r^2 \dv{r} \right) } R=
\frac{1}{r} \dv{t} r^2 \frac{1}{r} \dv{t} R
=
\frac{1}{r} \dv{t} r\dv{t} R
= \frac{1}{r} \left( r \ddot{R} + r \dot{R} \right) = \lambda R
\end{equation}
\begin{equation}
\ddot{R} + \dot{R} - \lambda R = 0
\end{equation}
目前\footnote{后面可知,只有当\(l\)为整数时,才能截断}\(\lambda \in \RR\),不妨设\(\lambda = l(l+1), l \in \RR\),
解得
\begin{subequations}
  \begin{align}
  &R_1 = C_1 e^{lt} = C_1 r^l \\
  &R_2 = C_2 e^{(-l-1)t} = C_2 r^{-l-1}
\end{align}
\end{subequations}
\begin{equation*}
	\begin{aligned}
		P_l(x) &= \frac{1}{2^l} \sum_{m=0}^{[\frac{l}{2}]}
	(-1)^m C_l^m C_{2l - 2m}^{l} x^{l - 2m}\\
&= \frac{1}{2^l} \sum_{m=0}^{[\frac{l}{2}]}
	(-1)^m \frac{(2l - 2m)!}{m! (l-m)! (l-2m)!} x^{l-2m}
	\end{aligned}
\end{equation*}
其中\([\ ]\)代表向下取整
\begin{equation*}
	\left[\frac{l}{2}\right] = 
	\begin{cases}
		\frac{l}{2} & \qq{\(l\) is even } \\
		\frac{l-1}{2} & \qq{\(l\) is odd }
	\end{cases}
\end{equation*}

\begin{align*}
	P_0 = 1 \\
	P_1 = x \\
	P_2 = \frac{1}{2}(3x^2 - 1) \\
	P_3 = \frac{1}{2}(5x^3 - 3x)
\end{align*}





%%% vim: set ts=2 sts=2 sw=2 isk+=\: et cc=+1 formatoptions+=mM:

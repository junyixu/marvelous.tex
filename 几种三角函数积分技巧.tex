% !TEX root = main.tex
% 几种三角函数积分技巧

\section{几种三角函数积分技巧}%
\begin{itemize}
	\item 万能代换:令\(t=\tan \frac{x}{2}\)
	\item \( R( -\sin \theta) = - R(\sin \theta) \),令\(u = \cos \theta\),凑\(\dd \cos \theta\)
	\item \( R( -\cos \theta) = - R(\cos \theta) \),令\(u = \sin \theta\),凑\(\dd \sin \theta\)
	\item \( R( -\sin \theta, -\cos \theta) = R(\sin \theta, \cos \theta) \),令\(u = \tan \theta\),凑\(\dd \tan \theta\)
\end{itemize}

\verb+j-j+耦合一般不会考,常考的是 \verb+S-L+耦合
测试

任何一个长度不变的矢量
在一个旋转系中的速度都可以写成
\begin{equation*}
	\dot{\vb{L}} = \va{\omega} \cross \vb{L}
\end{equation*}
\(\dot{\vb{L}}\)依然是个长度不变的矢量,
它的速度,也就是原先\(\vb{L}\)的加速度可以写成
\begin{equation*}
	\ddot{\vb{L}} = \va{\omega} \cross ( \va{\omega} \cross \vb{L})
\end{equation*}
根据矢量规则,可以写成:
\begin{equation*}
	\begin{aligned}
		\ddot{\vb{L}} &= bac - cab \\
									&= \va{\omega} ( \va{\omega} \vdot \vb{L} ) - \omega^2 \vb{L}
	\end{aligned}
\end{equation*}
\begin{equation*}
	\curl(\lambda \vb{A}) = \lambda \curl \vb{A} + \vb{A} \cross \grad \lambda
\end{equation*}
\begin{equation*}
	\div(\lambda \vb{A}) = \lambda \div \vb{A} + \vb{A} \vdot \grad \lambda
\end{equation*}

圆锥曲线极坐标
\begin{equation*}
	r(\phi) = \frac{p}{1 + e \cos \phi} 
\end{equation*}
对于椭圆有\(p = \frac{b^2}{a} \qq{and} e = \frac{c}{a}\)

椭圆的面积\(\pi ab\)



\begin{equation*}
	\begin{aligned}
		S &= \frac{1}{2}\int r^2 \dd \phi \\
&= \frac{1}{2}\int \left(\frac{p}{1 + e \cos \phi} \right)^2 \dd \phi
	\end{aligned}
\end{equation*}
\begin{tabular}{l}
\(\cos\phi \to \frac{z + z^{-1}}{2}, z=e^{i \phi} \) \\
\(\dd \phi \to \frac{\dd z}{i z}\)
\end{tabular}	
\begin{equation*}
	\begin{aligned}
		S &= \frac{1}{2}\oint_{\abs{z}=1} \left(\frac{p}{1 + e \left(\frac{z + z^{-1}}{2}\right)} \right)^2 \frac{\dd z}{i z}\\
			&= \frac{1}{2} \oint_{\abs{z}=1} f(z) \frac{1}{i} \dd z \\
			&= \pi \sum_i \Res(f, b_i)
	\end{aligned}
\end{equation*}

%%% vim: set ts=2 sts=2 sw=2 isk+=\: et cc=+1 formatoptions+=mM:

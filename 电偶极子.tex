% !TEX root = main.tex
% 电偶极子

\section{电偶极子}%
正负点电荷起初都在原点,
对于正电荷,场点向下移动,
等效于源点向上移动\(-\Delta \vb{r} = \Delta \vb{r}'\)
\begin{equation*}
	\Delta \phi_+=\phi'_+ - \phi_+ =
	\grad \phi_+ \vdot (-\Delta \vb{r})
	=\grad \phi_+ \vdot \Delta \vb{r}'
\end{equation*}
\begin{equation*}
	\phi'_+ = \phi_+ +
	\grad \phi_+ \vdot (-\Delta \vb{r})
\end{equation*}
对于负电荷,场点向上移动,等效于源点向下移动
\(\Delta \vb{r} = - \Delta \vb{r}'\)
\begin{equation*}
	\phi'_- =
	\phi_- + \grad \phi_- \vdot \Delta \vb{r}
\end{equation*}
向上移动的正点电荷与向下移动的负点电荷的线性叠加
\begin{equation*}
	\phi_+' + \phi_-' = 
	\grad \phi_+ \vdot 2 \Delta \vb{r}'
	=
	\grad \phi_+ \vdot \vb{d}
	=
	q \vb{d}  \vdot \grad \frac{1}{r}
\end{equation*}
记 \(q\vb{d}\) 为 \(\vb{p}\),电偶极子的位置就是正负电荷的中间,原点处电偶极子的势函数:
\begin{equation*}
	\phi = \phi_+' + \phi_-' = 
	-\vb{p} \vdot \grad \frac{1}{r}
\end{equation*}
若电偶极子不在原点:
\begin{equation*}
	\phi =
	- \vb{p} \vdot \grad \frac{1}{R}
\end{equation*}
注意到\(\grad \frac{1}{R} = - \frac{\vu{R}}{R^2}\)
\begin{equation*}
	\phi =
	 \frac{\vb{p} \vdot \vu{R}}{R^2}
\end{equation*}
补全系数
\begin{equation}
	\phi =
  \frac{1}{4 \pi \epsilon_0}
	 \frac{\vb{p} \vdot \vu{R}}{R^2}
\end{equation}
电偶极子电场:
\begin{equation}
  \vb{E} = \grad \phi= \frac{1}{4 \pi \epsilon_0} \left( \frac{\vb{p}}{R^3} - \frac{3\vb{p \vdot \vb{R}}}{R^5} \vb{R}\right)
\end{equation}
磁偶极子磁场
\begin{equation}
  \vb{B} = \curl \vb{A}= \frac{\mu_0}{4 \pi} \left( \frac{\vb{m}}{R^3} - \frac{3\vb{m \vdot \vb{R}}}{R^5} \vb{R}\right) 
\end{equation}
多个电偶极子势的线性叠加
\begin{equation*}
	\begin{aligned}
	  \sum_i \phi_i(\vb{r}) &=
	\sum_i {\vb{p}_i \vdot }{\grad \frac{1}{R_i}} \\
  &=
	 \int \dd \tau' \vb{P}  \vdot \grad \frac{1}{R} \\
  &=
	 \int \dd \tau'\div (  \frac{\vb{P}}{R}) -
	 \int \dd \tau'\left(\div \vb{P} \right)   \frac{1}{R} \\
  &=
	 \oint \dd \vb{S}' \vdot\vb{P} \frac{1}{R} -
	 \int \dd \tau'\left(\div \vb{P} \right)   \frac{1}{R} \\
  &=
	 \oint \dd S'  \frac{\sigma_b}{R} +
	 \int \dd \tau'  \frac{\rho_b}{R} \\
	\end{aligned}
\end{equation*}


%%% vim: set ts=2 sts=2 sw=2 isk+=\: et cc=+1 formatoptions+=mM:

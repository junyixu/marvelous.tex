% !TEX root = main.tex
% 静电场唯一性定理

\section{静电场唯一性定理}%
\subsection{第一唯一性定理}%
边界条件给定,假设
\begin{equation*}
	\laplacian \phi = - \frac{\rho}{\epsilon_0}
\end{equation*}
有两个解\(\phi_1, \phi_2\),现证明\(\phi_1 = \phi_2\),设\(\phi_3 = \phi_2 - \phi_1\)。

内部:
\begin{equation*}
	\laplacian \phi_1 = - \frac{\rho}{\epsilon_0}
\end{equation*}
\begin{equation*}
	\laplacian \phi_2 = - \frac{\rho}{\epsilon_0}
\end{equation*}
\begin{equation*}
	\laplacian \phi_3 = 0
\end{equation*}
边界上:
\begin{equation}
\phi_3 = \phi_2 - \phi_1 = 0
\end{equation}
\begin{equation*}
	\pdv{\phi_3}{r}
	=
	\pdv{\phi_2}{r}
	-
	\pdv{\phi_1}{r}
	=0
\end{equation*}

利用矢量恒等式:
\begin{equation*}
	\div(\phi \grad \phi) 
	= 
	(\grad \phi) \vdot (\grad \phi) 
	+
	\phi \laplacian \phi
	= 
	(\grad \phi)^2
	+
	\phi \laplacian \phi
\end{equation*}

再利用散度定理:
\begin{equation*}
	\oint \phi_3 \grad \phi_3 \vdot \dd \vb{S} 
	= 
	\int
	\left( 
	(\grad \phi_3)^2
	+
	\phi_3 \laplacian \phi_3\right)
	\dd V
\end{equation*}

由于\(\laplacian \phi_3=0\),所以
\begin{equation*}
	\oint \phi_3 \grad \phi_3 \vdot \dd \vb{S} 
	= 
	\int
	\left(\grad \phi_3\right)^2
	\dd V
\end{equation*}
由于边界处\(\phi_3 = 0\)且\(\grad \phi_3 = 0\)
\begin{equation*}
0 =
	\int
	\left(\grad \phi_3\right)^2
	\dd V
\end{equation*}
由于\(	\left(\grad \phi_3\right)^2 \geq 0\),所以
\begin{equation*}
	\grad \phi_3 = 0
\end{equation*}
所以\(\phi_3\)为常数,又由于\(\phi_3\)在边界处为零,若\(\phi_3\)连续
\begin{equation*}
	\phi_3 = 0
\end{equation*}
即
\begin{equation*}
 \phi_2 = \phi_1
\end{equation*}

\subsection{第二唯一性定理}%

\begin{equation*}
	\div(\grad \phi_1 )= \frac{\rho}{\epsilon}
\end{equation*}
\begin{equation*}
	\div(\grad \phi_2 )= \frac{\rho}{\epsilon}
\end{equation*}
\begin{equation*}
	\div\left(\grad (\phi_2 - \phi_1) \right)=0
\end{equation*}

导体的边界上:
\begin{equation*}
	\oint (\grad \phi_1 ) \vdot \dd \vb{S}= \frac{Q}{\epsilon}
\end{equation*}
\begin{equation*}
	\oint (\grad \phi_2 ) \vdot \dd \vb{S}= \frac{Q}{\epsilon}
\end{equation*}
\begin{equation}
	\oint (\grad \phi_3 ) \vdot \dd \vb{S}=0
	\label{eq:ophi3}
\end{equation}

矢量恒等式:
\begin{equation*}
	\div(\phi \grad \phi) 
	= 
	(\grad \phi) \vdot (\grad \phi) 
	+
	\phi \laplacian \phi
	= 
	(\grad \phi)^2
	+
	\phi \laplacian \phi
\end{equation*}

再利用散度定理:
\begin{equation*}
	\oint \phi_3 \grad \phi_3 \vdot \dd \vb{S} 
	= 
	\int
	\left( 
	(\grad \phi_3)^2
	+
	\phi_3 \laplacian \phi_3\right)
	\dd V
\end{equation*}

由于\(\laplacian \phi_3=0\),所以
\begin{equation*}
	\oint \phi_3 \grad \phi_3 \vdot \dd \vb{S} 
	= 
	\int
	\left(\grad \phi_3\right)^2
	\dd V
\end{equation*}
由于导体是等势体,\(\phi_3\)在边界处是个常数
\begin{equation*}
\phi_3 	\oint \grad \phi_3 \vdot \dd \vb{S} 
	= 
	\int
	\left(\grad \phi_3\right)^2
	\dd V
\end{equation*}
Eq.~\eqref{eq:ophi3} 
\begin{equation*}
	0
	= 
	\int
	\left(\grad \phi_3\right)^2
	\dd V
\end{equation*}
由于\(	\left(\grad \phi_3\right)^2 \geq 0\),所以
\begin{equation*}
	\grad \phi_3 = 0
\end{equation*}




%%% vim: set ts=2 sts=2 sw=2 isk+=\: et cc=+1 formatoptions+=mM:

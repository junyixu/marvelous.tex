\section{极限}%
\begin{exercise}
	\begin{equation*}
		\lim_{\substack{x \to 0 \\ y \to 0}} \frac{xy}{\sqrt{x^2 + y^2}} = 0
	\end{equation*}
\end{exercise}
\begin{proof}
	\begin{equation}
		\left|  \frac{xy}{\sqrt{x^2 + y^2}} \right|
		\leq
		\frac{(x^2+y^2) / 2}{ \sqrt{x^2 + y^2} }
		=
		\frac{\sqrt{x^2 + y^2}}{2}
	\end{equation}
	\mn{
		\(
		\left|xy\right| \leq (x^2 + y^2)/2
		\)
	}
	即:
	\begin{equation*}
		0
		\leq
		\left|  \frac{xy}{\sqrt{x^2 + y^2}} \right|
		\leq
		\frac{\sqrt{x^2 + y^2}}{2}
	\end{equation*}
	\marginnote{
		\( \begin{aligned}
			 & \forall \varepsilon > 0                                                          \\
			 & \exists \delta = 2\varepsilon \qc 0 < \rho <\delta                               \\
			 & \st \frac{\sqrt{x^2 + y^2}}{2} = \frac{\rho}{2} < \frac{\delta}{2} = \varepsilon
		\end{aligned} \)
	}
	\begin{equation}
		0
		\leq
		\lim_{\substack{x \to 0 \\ y \to 0}}
		\left|  \frac{xy}{\sqrt{x^2 + y^2}} \right|
		\leq
		\lim_{\substack{x \to 0 \\ y \to 0}}
		\frac{\sqrt{x^2 + y^2}}{2}
		=
		0
	\end{equation}

	\( 		\therefore \)
	\begin{equation}
		\lim_{\substack{x \to 0 \\ y \to 0}} \frac{xy}{\sqrt{x^2 + y^2}} = 0
	\end{equation}
\end{proof}
\begin{exercise}
	设
	\begin{equation*}
		f(x) =
		\begin{cases}
			\frac{(x+y) \sin xy }{x^2 + y^2} \qc x^2 + y^2 \neq 0 \\
			0 \qc x^2 + y^2 = 0
		\end{cases}
	\end{equation*}
	证明:
	\( f(x,y) \) 在 \( (0,0) \) 连续但不可微
\end{exercise}
\begin{proof}
	由于
	\begin{equation*}
		\left| \frac{(x+y) \sin xy }{x^2 + y^2}  \right|
		\leq
		\left| \frac{(x+y) xy}{2 x y}  \right|
		\leq
		\frac{\left| x + y \right|}{2}
		\leq
		\frac{\left| x \right|}{2}
		+
		\frac{\left| y \right|}{2}
	\end{equation*}
	\( \forall \varepsilon > 0 , \exists \delta =  \varepsilon \)  当 \( \left| x \right| \bra{ \delta , \left| y \right| < \delta } \) 时
	\begin{equation*}
		\left| f(x,y) - f(0,0) \right|
		=
		\left| \frac{(x+y) \sin xy }{x^2 + y^2}  \right|
		<
		\frac{\left| x \right|}{2}
		+
		\frac{\left| y \right|}{2}
		<
		\varepsilon
	\end{equation*}
	即:
	\begin{equation}
		\lim_{\substack{ x \to 0 \\ y \to 0 } } f(x,y)
		=
		f(0,0)
		=0
	\end{equation}
	故, \( f(x,y) \) 在点 \( (0,0) \) 处连续, 下证 \( f(x,y) \)在点\(  \)
	\begin{align*}
		f_x'(0,0) & = \lim_{x \to 0} \frac{ f(x,0) - f(0,0) }{ x - 0 } \\
		          & = \frac{ \frac{x \vdot 0}{x^2} }{x}                \\
		          & = \frac{0}{x^3} = 0
	\end{align*}
	同理 \( f'_y(0,0) = 0 \)
	令
	\begin{equation*}
		\Delta \omega = f(x,y) - f(0,0) - f'_x(0,0) \Delta x - f'_y(0,0) \Delta y =  \frac{(x+y) \sin xy }{x^2 + y^2}
	\end{equation*}
	而\marginnote{
		\begin{equation*}
			\begin{aligned}
				 & \lim_{\substack{ \rho \to 0           \\ y = kx }}
				\frac{ k+1 }{(k^2 +1)^{\frac{3}{2}}}
				\cdot
				\frac{ \sin k x^2 }{x^2}                 \\
				 & =
				\lim_{\substack{ \rho \to 0              \\ y = kx }}
				\frac{ k+1 }{(k^2 +1)^{\frac{3}{2}}}
				\lim_{\substack{ \rho \to 0              \\ y = kx }}
				\frac{ k \sin \red{k x^2} }{\red{k x^2}} \\
				 & =
				\frac{(k+1)k}{ (k^2 + 1)^{\frac{3}{2}} } \quad \text{与\( k \)有关}
			\end{aligned}
		\end{equation*}
	}
	\begin{equation*}
		\begin{aligned}
			\lim_{ \rho \to 0 }
			\frac{\Delta \omega}{\rho}
			 & =
			\lim_{ \rho \to 0 }
			\frac{(x+y) \sin xy }{(x^2 + y^2)^{\frac{3}{2}}} \\
			 & =
			\lim_{\substack{ \rho \to 0                      \\ y = kx }}
			\frac{(x+y) \sin xy }{(x^2 + y^2)^{\frac{3}{2}}} \\
			 & =
			\lim_{\substack{ \rho \to 0                      \\ y = kx }}
			\frac{ k+1 }{(k^2 +1)^{\frac{3}{2}}}
			\cdot
			\frac{ \sin k x^2 }{x^2}                         \\
			 & =
			\frac{(k+1)k}{ (k^2 + 1)^{\frac{3}{2}} } \quad \text{与\( k \)有关}
		\end{aligned}
	\end{equation*}
\end{proof}

\begin{exercise}
	将 \( f(x) = \arctan \frac{1+x}{1-x} \)展成\( x \)的幂级数。
	\begin{margintable}
		\begin{tabularx}{\marginparwidth}{|X}
			Section~. Motivation        \\
			Section~. Required Packages \\
			Section~. Margins           \\
		\end{tabularx}
	\end{margintable}
	解:
	\begin{align*}
		 & f'(x) = \frac{1}{1 + (\frac{1+x}{1-x})^2} \vdot \frac{1 - x - (1+x) (-1)}{ (1-x)^2 }                        \\
		 & = \frac{2}{2+2x^2} = \frac{1}{1+x^2} = \sum_{n=0}^{\infty)} (-1)^n x^{2n} \quad \text{收敛半径}(-1 < x < 1)
	\end{align*}
	\marginnote{
		令
		\( x = -1 \)
		\begin{gather*}
			0 = \frac{\pi}{4} + \sum_{n=0}^\infty (-1)^{n+1} \frac{1}{2n+1} \\
			-\sum_{n=0}^\infty (-1)^{n+1} \frac{1}{2n+1}
			=
			\frac{\pi}{4} \\
			4 \sum_{n=0}^\infty (-1)^{n+2} \frac{1}{2n+1}
			=
			\pi \\
			4 \sum_{n=0}^\infty (-1)^{n} \frac{1}{2n+1}
			=
			\pi
		\end{gather*}
		令 \( n=m-1 \)
		\begin{equation*}
			\begin{aligned}
				\pi
				 & =
				4 \sum_{m=1}^\infty (-1)^{m-1} \frac{1}{2m-1} \\
				 & =
				4 \sum_{n=1}^\infty (-1)^{n-1} \frac{1}{2n-1}
			\end{aligned}
		\end{equation*}
		误差(莱布尼兹判别法)
		\begin{equation*}
			\left| R_n \right| \leq \frac{4}{\underbrace{2n+1}_{u_{n+1}}}
		\end{equation*}
	}
	\begin{gather*}
		\int_0^x f'(x) \dd x = \sum_{n=0}^\infty (-1)^n \int_0^x x^{2n} \dd x  \\
		f(x) - f(0) =
		\sum_{n=0}^\infty (-1)^n \frac{x^{2n+1}}{2n+1}\\
		f(x) - f(0) = \sum_{n=0}^\infty (-1)^n \frac{x^{2n+1}}{2n+1}\\
		f(0) = \arctan 1 = \frac{\pi}{4}\\
		\therefore f(x) = \arctan \frac{1+x}{1-x} = \frac{\pi}{4} +
		\sum_{n=0}^\infty (-1)^n \frac{x^{2n+1}}{2n+1}
	\end{gather*}
\end{exercise}


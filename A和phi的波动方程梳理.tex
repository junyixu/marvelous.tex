% !TEX root = main.tex
% A和phi的波动方程梳理

\section{A和phi的波动方程梳理}

写出 Maxwell Equations 后

根据两个重要矢量恒等式等于0,通过 Maxwell Equations 的无源部分,定义出矢势和标势
\begin{equation}
	\begin{aligned}
	\vb{B} = \curl \vb{A} \\
	\vb{E} =  - \grad \phi - \pdv{\vb{A}}{t}
	\label{eq:def_of_A_and_phi}
	\end{aligned}
\end{equation}

把这两个定义代入 Maxwell Equations 的有源部分,利用 Lorentz Gauge,写出两个波动方程
\begin{equation}
	\begin{aligned}
		\da \vb{A} = - \mu_{0}  \vb{j} \\
	\da \phi = - \frac{\rho}{\epsilon_0}
	\label{eq:wave_eq_of_A_and_phi}
	\end{aligned}
\end{equation}
通过格林函数法,解方程,引入推迟势的概念
得到他们的解

真空中有:
\begin{equation}
	\begin{aligned}
		\da \vb{A} = 0 \\
	\da \phi =0
	\label{eq:wave_eq_of_A_and_phi_without_source}
	\end{aligned}
\end{equation}
同时对\(\vb{A}\)取旋度后代入Eq.~\eqref{eq:wave_eq_of_A_and_phi_without_source}交换算符的符号,再代入Eq.~\eqref{eq:def_of_A_and_phi} 的第二个式子和Eq.~\eqref{eq:wave_eq_of_A_and_phi_without_source} 的第二个式子,即,把Eq.~\eqref{eq:def_of_A_and_phi} 代入Eq.~\eqref{eq:wave_eq_of_A_and_phi_without_source} 
可以写出\(\vb{E}\)和\(\vb{B}\)的波动方程
\begin{equation}
	\begin{aligned}
		\Da{\vb{E}} = 0 \\
		\Da{\vb{B}} = 0
	\end{aligned}
\end{equation}




%%% vim: set ts=2 sts=2 sw=2 isk+=\: et cc=+1 formatoptions+=mM:

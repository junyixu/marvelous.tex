%! TEX root = debug.tex
% 统计力学
配分函数与热力学量
\begin{equation}
  Z(T,V) = \int e^{-\beta \varepsilon} \dd{\omega} 
  \label{eq:partion_function}
\end{equation}
\begin{equation*}
  U = - N \pdv{\beta} \ln Z
\end{equation*}
\begin{equation*}
  S  = N k T ( \ln Z - \beta \pdv{\beta} \ln Z )
\end{equation*}
\begin{equation}
  F = U -TS = - NkT \ln Z
  \label{eq:FreeEnergy}
\end{equation}
\begin{theorem}[马休定理]
  给出\(F(T,V)\)就能求出一切热力学量。
  \label{th:马休定理}
\end{theorem}
根据定理\ref{th:马休定理}和Eq.~\eqref{eq:FreeEnergy}和Eq.~\eqref{eq:partion_function}得出:给出\(Z(T,V)\)就能求出一切热力学量

\begin{definition}[广义力]
  \begin{equation}
  p = - \sum_l a_l \pdv{\varepsilon_l}{V}
  \end{equation}
  \label{def:广义力}
\end{definition}
对于光子气体,处于第\(l\)个相格的光子的能量为
\begin{equation}
  \varepsilon_l  = c p_l = c \frac{2\pi \hbar}{L} \sqrt{n_x^2 + n_y^2 + n_z^2}
  \label{eq:光子气体能量}
\end{equation}
其中处于第\(l\)个相格的光子的动量大小为\sn{由箱归一化周期性边界条件得到}
\begin{equation*}
p_l = \frac{2\pi \hbar}{L} \sqrt{n_x^2 + n_y^2 + n_z^2}
\end{equation*}
把\cref{eq:光子气体能量}代入\cref{def:广义力}得
\begin{equation}
  p = \frac{1}{3} \sum_l a_l \varepsilon_l / V = \frac{u}{3}
\end{equation}
其中\(u\)为能量密度

%%% vim: set ts=2 sts=2 sw=2 isk+=\: et cc=+1 formatoptions+=mM:

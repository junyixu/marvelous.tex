% !TEX root = main.tex
% LaplaceTransform


\section{Laplace Transform}%
\subsection{由幂级数引入拉普拉斯变换}%
\begin{equation*}
	\sum_{n=1}^{\infty} a_n = A(x)
\end{equation*}

\begin{equation*}
	\sum_{n=1}^{\infty} a(n) = A(x)
\end{equation*}
If time continues
\begin{equation*}
	\int_{0}^{\infty} \red{a(t)} x^t \dd t = \blue{A(x)}  \qc \text{收敛条件:} 0<x<1
\end{equation*}
根据恒等式
\begin{gather*}
	x = e^{\ln x}\\
	x^t = e^{t \ln x}
\end{gather*}
\(-s = \ln x < 0 \implies s>0\)
\begin{remark}
	为什么要令 \(-s = \ln x\)?

	因为 \(\ln x\) 自然地限制了 \(x > 0\),只要再加一个条件\(\ln x < 0\) 就又限制了 \(x<1\)。
\end{remark}

\begin{equation*}
	\int_{0}^{\infty} \red{f(t)} e^{-st} \dd t = \blue{F(s)}  \qc \text{收敛条件:} s > 0
\end{equation*}

线性性质:
\begin{itemize}
	\item 加和:\(\La[f+g] = \La[f]+ \La[g]\)
	\item 数乘: \(\La[cf] = c \La[f]\)
\end{itemize}


% TODO
“基”相当于“自变量”,由于基改变了,线性变换称为变换

\begin{itemize}
	\item \(f(t)\) transform \(F(s)\)
	\item \(f(t) \) operator \(g(t)\)
\end{itemize}

\begin{equation*}
	\La[t^n] = \frac{n}{s} \La[t^{n-1}] = \frac{n!}{s^{n+1}}
\end{equation*}
	
\begin{equation*}
	\La[f'(t)] = s\La[f(t)] - f(0)
\end{equation*}
\begin{equation*}
	\La[f''(t)] = s\La[f'(t)] - f'(0)
\end{equation*}

Use Laplace transform to solve linear differential equations with constant coefficients.

Laplace Transform, exist?

Condition: \(f(t)\) doesn't grow so rapidly. ``exponential type''



%%% vim: set ts=2 sts=2 sw=2 isk+=\: et cc=+1 formatoptions+=mM:

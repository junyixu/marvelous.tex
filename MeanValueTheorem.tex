% MeanValueTheorem


\section{Mean Value Theorem}%
% 中值定理
Fermat Lemma
\( \implies \)
Rolle Theorem
\( \implies \)
Lagrange's Mean Value Theorem
\( \implies \)
Cauchy Mean Value Theorem

\subsection{Fermat Lemma}%
\label{sub:Fermat_Lemma}

\begin{equation*}
\frac{f(x + \Delta x) - f(x)}{\Delta x}
\end{equation*}

\begin{enumerate}
	\item Minimum: 
	\begin{enumerate}
		\item If \( \Delta x > 0 \), \( f(x + \Delta x) - f(x) > 0\) , 
\( \frac{f(x + \Delta x) - f(x)}{\Delta x} >0 \)
		\item If \( \Delta x < 0 \), \( f(x + \Delta x) - f(x) < 0\) 
\( \frac{f(x + \Delta x) - f(x)}{\Delta x} <0 \)
	\end{enumerate}
	\item Maximum 
\begin{enumerate}
		\item If \( \Delta x > 0 \), \( f(x + \Delta x) - f(x) < 0\) , 
\( \frac{f(x + \Delta x) - f(x)}{\Delta x} <0 \)
		\item If \( \Delta x < 0 \), \( f(x + \Delta x) - f(x) > 0\) 
\( \frac{f(x + \Delta x) - f(x)}{\Delta x} >0 \)
	\end{enumerate}
\end{enumerate}
\begin{equation*}
	h(\xi) = \frac{f(x + \xi) - f(x)}{\xi}
\end{equation*}
Zero point theorem:
\begin{enumerate}
	\item If \( h(\xi^-) < 0 \) and \( h(\xi^+) > 0 \), then \( h(\xi) = 0 \)
	\item If \( h(\xi^-) > 0 \) and \( h(\xi^+) < 0 \), then \( h(\xi) = 0 \)
\end{enumerate}

\subsection{Rolle Theorem}%
\label{sub:Rolle_Theorem}

Give an arbitrary curve. Two possibilities:
\begin{enumerate}
	\item Extremum is one of the endpoints of the curve. The curve is a straight line. \( \forall \xi \in (x_1, x_2), f'(\xi) = 0 \).
	\item Extremum in the middle of the curve. There must be a point, say \( \xi \), \( s.t. f'(\xi) = 0 \), according to \cref{sub:Fermat_Lemma}.
\end{enumerate}

So if 
\begin{enumerate}
	\item continuous in \( [x_1, x_2] \) 
	\item differentiable in \( (x_1, x_2) \) 
	\item \( f(x_1) = 0, f(x_2) = 0 \),
\end{enumerate}
\( \exists \xi \in (x_1, x_2) s.t. f'(\xi) = 0 \).
\subsection{Lagrange's Mean Value Theorem}%
\sidenote{
\begin{equation*}
	\Delta y = y'(\xi) \Delta x \qc \xi \in (x, x + \Delta x)
\end{equation*}
If \( \Delta x \to 0 \), then \( \xi \to x \)
\begin{align*}
	\Delta y &= y'(\xi) \\ 
					 &\approx  y'(x) \Delta x \qc \xi \in (x, x + \Delta x)
\end{align*}
\( \dd x = \Delta x, \Delta x \to 0 \)
\begin{equation*}
	\dd y = y'(x) \dd x 
\end{equation*}
% which is fundamental theorem of Calculus.
}
Rotate the x y coordinate system according to Section~\ref{sub:Rolle_Theorem}.
\begin{equation}
	\frac{y_2 - y_1}{x_2 - x_1} = y'(\xi) \qc \xi \in (x_1, x_2) %TODO []  or ()
	\label{eq:LagrangeTheorem}
\end{equation}


\subsection{Cauchy Mean Value Theorem}%
	According to Eq.~\eqref{eq:LagrangeTheorem} 
\begin{equation*}
\begin{aligned}
	\frac{f_2 - f_1}{g_2 - g_1} = f'(\xi) &= \eval{\dv{f}{g}}_{t = \xi} \\ %TODO []  or () 
	&= \eval{\left(\dv{f}{t} \vdot
	\dv{t}{g}\right)}_{t = \xi} 
	= \frac{f'(\xi)}{g'(\xi)}
	\qc \xi \in (t_1,t_2)
\end{aligned}
\end{equation*}
i.e.
\begin{equation}
	\frac{f(t_2) - f(t_1)}{g(t_2) - g(t_1)} = 
	= \frac{f'(\xi)}{g'(\xi)}
	\qc \xi \in (t_1,t_2)
\end{equation}

%%% vim: set ts=2 sts=2 sw=2 isk+=\: et cc=+1 formatoptions+=mM:

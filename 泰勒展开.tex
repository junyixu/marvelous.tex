% !TEX root = main.tex
% 泰勒展开

乘积顺序不能换,但直积和点积,结合律还是能用的(和点积与叉积不同,因为矢量叉乘得到矢量,矢量点积得到标量)
\begin{equation*}
	\int \dd V \div(\vb{j}\vb{x}) 
	= \oint \dd \vb{S} \vdot (\vb{j}\vb{x}) 
	= \oint (\dd \vb{S} \vdot \vb{j})\vb{x} 
\end{equation*}
\(- \vb{x'}\) 为小量
\begin{equation*}
	\begin{aligned}
	\frac{1}{ \abs{\vb{x} - \vb{x'}} }
	&= \frac{1}{0!} \frac{1}{\abs{\vb{x}}}
	+ \frac{1}{1!} (\vb{-x'}) \vdot \grad \frac{1}{\abs{\vb{x}}}
	+ \frac{1}{2!} (\vb{-x'}) (\vb{-x'}) : \grad\grad \frac{1}{\abs{\vb{x}}} \\
	&= \frac{1}{0!} \frac{1}{\abs{\vb{x}}}
	- \frac{1}{1!} \vb{x'} \vdot  \grad \frac{1}{\abs{\vb{x}}}
	+ \frac{1}{2!} \vb{x'}\vb{x'} : \grad\grad \frac{1}{\abs{\vb{x}}}
	\end{aligned}
\end{equation*}
\begin{equation*}
	\Delta \phi = \grad \phi \vdot \Delta \vb{x}
\end{equation*}
\begin{equation*}
	 \phi(\vb{x} + \Delta \vb{x}) - \phi(\vb{x})  = \grad \phi \vdot \Delta \vb{x}
\end{equation*}
\begin{equation*}
	 \phi(\vb{x} + \Delta \vb{x}) = \phi(\vb{x})  + \grad \phi \vdot \Delta \vb{x}
\end{equation*}
\begin{equation*}
	 \phi(\vb{x} + \Delta \vb{x}) = \phi(\vb{x})  +  \Delta \vb{x}\vdot \grad \phi
\end{equation*}
\begin{equation*}
	 \phi(\vb{x} + \Delta \vb{x}) = \frac{1}{0!} \phi(\vb{x})  + \frac{1}{1!} \Delta \vb{x}\vdot \grad \phi
\end{equation*}
\begin{equation*}
	 \phi(\vb{x} + \Delta \vb{x}) = \frac{1}{0!} \phi(\vb{x})  +
	 \frac{1}{1!} \Delta \vb{x}\vdot \grad \phi(\vb{x})
	 + \frac{1}{2!} \Delta \vb{x} \Delta \vb{x} : \grad \grad \phi(\vb{x}) 
\end{equation*}




%%% vim: set ts=2 sts=2 sw=2 isk+=\: et cc=+1 formatoptions+=mM:

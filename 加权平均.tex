% !TEX root = main.tex
% 加权平均

\section{加权平均}%

一维:\(x\)轴被无限细分成小块,第\(i\)块为\(\Delta x_i\),这第\(i\)小块内有\(\Delta m_i\)落入其中
\begin{tabular}{ccc}
	\(\Delta x_1\) &   对应 & \(\Delta m_1\) \\
	\(\Delta x_2\) &   对应 & \(\Delta m_2\) \\
	\(\Delta x_3\) &   对应 & \(\Delta m_3\) \\
	\(\Delta x_4\) &   对应 & \(\Delta m_4\) \\
	…… & 对应 & …… \\
	\(\Delta x_n\) &   对应 & \(\Delta m_n\)
\end{tabular}

总质量是对所有质量块的求和:
\begin{equation*}
	m = \sum_i \Delta  m_i
\end{equation*}
质心
\begin{equation*}
	x_c =
	\frac{ \sum_i \Delta x_i \Delta m_i }{\sum_i \Delta m_i}
\end{equation*}
定义 \(\rho(x)\) 为线密度
, 在全空间积分
\begin{align*}
	m &= \int \dd m \\
	x_c  &= \frac{1}{m} \int x \dd m \\
	&= \frac{1}{m} \int_{-\infty}^{\infty} x \dv{m}{x} \dd x \\
	&= \frac{1}{m} \int_{-\infty}^{\infty} x \rho(x) \dd x \\
\end{align*}


平均时间
\begin{equation*}
	\tau =
	\frac{ \sum_i \Delta n_i \Delta t_i }{\sum_i \Delta n_i}
\end{equation*}
\begin{align*}
	- \dd n = n \lambda \dd t \\
	n = n_0 e^{-\lambda t}
\end{align*}
\begin{align*}
	\tau &=
	\frac{1}{n_0} \int^0_{n_0} t \dd n \\
			 &=
	\frac{1}{n_0} \int_0^{n_0} t \red{( - \dd n )} \\
			 &=
	\frac{1}{n_0} \int_0^{\infty} t \red{(n \lambda \dd t)} \\
			 &=
	\frac{1}{n_0} \int_0^{\infty} t \red{(n_0 e^{-\lambda t} \lambda \dd t)} \\
\end{align*}

平均能量

有\(a_l\) 个粒子能量为 \(\varepsilon_l\)
\begin{tabular}{ccc}
	\( a_1\) &   个粒子能量为 & \( \varepsilon_1\) \\
	\( a_2\) &   个粒子能量为 & \( \varepsilon_2\) \\
	\( a_3\) &   个粒子能量为 & \( \varepsilon_3\) \\
	\( a_4\) &   个粒子能量为 & \( \varepsilon_4\) \\
	…… & 个粒子能量为& …… \\
	\( a_n\) &   个粒子能量为 & \( \varepsilon_n\)
\end{tabular}
\begin{equation}
	\label{eq:planck_ave_energy}
	\overline{\varepsilon} = \frac{\sum_l a_l \varepsilon_l}{\sum_l a_l} = - \pdv{\ln Z}{\beta} = \frac{h \nu}{\exp(\frac{h\nu}{kT}) - 1}
\end{equation}





%%% vim: set ts=2 sts=2 sw=2 isk+=\: et cc=+1 formatoptions+=mM:

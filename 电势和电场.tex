% !TEX root = main.tex
% 电势和电场

\section{ \( \vb{E} = - \grad \varphi \) }%

在标量场\( \varphi \)中(\( \varphi \)是电势),将一单位电荷从 1 移到 2,整个过程是准静态的。假设从1到2的电势升高 \( \Delta \varphi \)。\sn{就算是降低也没有关系,那么 \( \Delta \varphi \)是负的就行了}
\begin{equation*}
	\Delta \varphi = \varphi(x_2) - \varphi(x_1) 
	= 
	\pdv{\varphi}{x} \Delta x
\end{equation*}
那么外力(非静电力)需要对此单位电荷做功\( \Delta W \) 使之电势升高
\begin{equation*}
	\Delta W = \Delta \varphi
\end{equation*}

在矢量场\( \vb{E} \)中(\( \vb{E} \)是电场强度),由于准静态过程,外力\( \vb{F} \)和静电力\( \vb{E} \)受力平衡:
\begin{equation*}
	\vb{F} + \vb{E}q = 0
\end{equation*}
在数值上
\begin{equation*}
	\vb{F} + \vb{E} = 0
\end{equation*}
由功的定义:
\begin{equation*}
	\Delta W = F \Delta x = -E_x \Delta x
\end{equation*}
可得
\begin{equation*}
	\pdv{\varphi}{x} = - E_x
\end{equation*}
三维情况
\begin{align*}
	\pdv{\varphi}{x} = - E_x \\
	\pdv{\varphi}{y} = - E_y \\
	\pdv{\varphi}{z} = - E_z
\end{align*}
\begin{equation}
	\grad \varphi = -\vb{E}
\end{equation}
综上,从点1到点2,外力做功,在标量场中只要计算一个减法
\begin{equation*}
 \frac{\Delta W}{q}	 = \varphi(\vb{r}_2) - \varphi(\vb{r}_1)
\end{equation*}
而在矢量场中需要计算三个积分
\begin{equation*}
	\frac{\Delta W}{q}	 =\int_{\vb{r}_1}^{\vb{r}_2} - \vb{E} \dd \vb{r}
\end{equation*}

\begin{proof}
	\begin{equation}
		\curl [ f(r) \vu{r} ]
		\label{eq:球对称的力}
	\end{equation}
\end{proof}
\begin{enumerate}
	\item 由库伦定律,点电荷\( q_i \)的电场\( \vb{E}_i \)满足式~\eqref{eq:球对称的力} 
	\item \( \curl \vb{E}_i = 0 \)
	\item 叠加\sn{叠加与坐标轴的选取无关}原理 
		\( \curl \left(\sum_i \vb{E}_i\right) = 0 \)
	\item \( \forall \vb{E}, \curl \vb{E} = 0 \). 
	\item 由\( \curl \grad \varphi \equiv 0 \)知,\( \varphi \)必然存在。
\end{enumerate}




%%% vim: set ts=2 sts=2 sw=2 isk+=\: et cc=+1 formatoptions+=mM:

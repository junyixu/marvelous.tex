\section{热扩散方程}%
如图~\ref{fig:热流与温度的关系} 可知
\begin{marginfigure}
	\includesvg[width=0.8\textwidth]{figures/热流与温度的关系.svg}
	\caption{热流与温度的关系。不管人为选定的\( x \)轴的方向如何,热量总是自发地从高温热源流向低温热源。}
	\label{fig:热流与温度的关系}
\end{marginfigure}
若\( T_2 - T_1 >0 \)那么热量从红线向蓝线流动\( \vb{h} \)的方向向左

设\( \Delta J \)
为从1到2传递的热量的大小,显然由于\( T_2>T_1 \),热量应该由2传到1,所以\( \Delta J \)为负值。

直觉上,\( \Delta J \)
\begin{itemize}
	\item 与厚度\( \Delta s  \)成负相关
	\item 与温差 \( \Delta T \)成正相关
	\item 与面积 \( \Delta A \)成正相关
	\item 与材料的性质,即热导率 \( \kappa \)有关
\end{itemize}
我们简单粗暴地写成反比和正比的形式(对于大多数金属材料确实近似如此)
\begin{equation*}
	\Delta J = \red{-} \kappa \Delta A  \Delta T / \Delta s
\end{equation*}
\sn{\red{这里的负号体现着热二定律}}
设 \( h = \frac{\Delta J}{\Delta A} \)为单位面积的传递的热量,称为热流
\begin{equation*}
	h = - \kappa \Delta T / \Delta s
\end{equation*}
\begin{equation*}
	h = - \kappa \frac{\dd T}{\dd s} \qq{或} h_x = - \kappa \dv{T}{x}
\end{equation*}
对于三维情况:
\begin{align*}
	h_x = - \kappa \pdv{T}{x} \\
	h_y = - \kappa \pdv{T}{y} \\
	h_z = - \kappa \pdv{T}{z}
\end{align*}
\begin{equation}
	\vb{h} = - \kappa \grad T
	\label{eq:热流}
\end{equation}

选取一小块体积 \( \Delta V \), 设 \( \Delta V \) 中含有热量 \( \Delta Q \), \( \Delta Q \) 的减少率等于流出去的热量(热流\( \vb{h} \)对该表面的环积分)
\begin{equation}
	-	\pdv{\Delta Q}{t} = \iint_{\partial (\Delta V)} \vb{h} \vdot \vb{n} \dd A = \div \vb{h} \Delta V
\end{equation}
\begin{equation*}
	-\pdv{t}(\frac{\Delta Q}{\Delta V}) = \div \vb{h}
\end{equation*}
设\( q = \frac{\Delta Q}{\Delta V} \) 为单位体积内含有的的热量
\begin{equation}
	-\pdv{q}{t} = \div \vb{h}
	\label{eq:热量连续性}
\end{equation}
\begin{equation}
	q = c_v T
	\label{eq:热容}
\end{equation}
把式~\eqref{eq:热流}和式~\eqref{eq:热容} 带入式~\eqref{eq:热量连续性}
\begin{equation}
	\pdv{T}{t} = \frac{\kappa}{c_v} \laplacian T
	\label{eq:热扩散方程}
\end{equation}
式~\eqref{eq:热扩散方程} 常被写为:
\begin{equation}
	\pdv{T}{t} =  D \laplacian T
\end{equation}
\( D \)称为扩散常数,这里等于\( \frac{\kappa}{c_v} \)。

\begin{itemize}
	\item Lagrangian 代表着全部的物理
	      \begin{itemize}
		      \item 物理系统本身
		            \begin{enumerate}
			            \item 物理系统本身
			            \item 物理规律对系统的支配
		            \end{enumerate}
		      \item 物理规律对系统的支配
	      \end{itemize}
	\item 三个守恒来自同一个原理:\( \var L = 0 \)
\end{itemize}


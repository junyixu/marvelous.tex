% 傅里叶变换

\section{傅里叶变换}
傅里叶级数
\begin{equation*}
	f(x) = \frac{a_0}{2} +
	\sum_{n=1}
	\left(
	a_n \cos(n x)  +
	b_n \sin(n x)
	\right)
\end{equation*}
其中
\begin{equation*}
	\begin{aligned}
		a_n = \frac{1}{\pi} \int_0^{2 \pi}  f(x) \cos{n x} d x \\
		b_n = \frac{1}{\pi} \int_0^{2 \pi}  f(x) \sin{n x} d x
	\end{aligned}
\end{equation*}
\begin{equation*}
	f(x) = \frac{a_0}{2} +
	\sum_{n=1}
	\left(
	a_n \cos{\frac{n \pi x}{L}}  +
	b_n \sin{\frac{n \pi x}{L}}
	\right)
\end{equation*}
其中
\begin{equation*}
	\begin{aligned}
		a_n = \frac{1}{\pi} \int_0^{2 \pi}  f(x) \cos{n x} d x \\
		b_n = \frac{1}{\pi} \int_0^{2 \pi}  f(x) \sin{n x} d x
	\end{aligned}
\end{equation*}
\begin{equation*}
	\begin{aligned}
		a_n = \frac{1}{L} \int_0^{2 L}  f(x) \cos(\frac{n \pi x}{L}) d x \\
		b_n = \frac{1}{L} \int_0^{2 L}  f(x) \sin(\frac{n \pi x}{L}) d x
	\end{aligned}
\end{equation*}
若\(f(t)\)是周期函数,则有
\begin{equation*}
	\int_0^{T}  f(t)  \dd t
	=
	\int_{0+x}^{T+x}  f(t)  \dd t
\end{equation*}
故
\begin{equation*}
	\begin{aligned}
		a_n = \frac{1}{L} \int_{-L}^{L}  f(x) \cos(\frac{n \pi x}{L}) d x \\
		b_n = \frac{1}{L} \int_{-L}^{ L}  f(x) \sin(\frac{n \pi x}{L}) d x
	\end{aligned}
\end{equation*}
\begin{equation}
	\begin{aligned}
		a_n = \frac{1}{L} \int_{-L}^{L}  f(t) \cos(\frac{n \pi t}{L}) d t \\
		b_n = \frac{1}{L} \int_{-L}^{ L}  f(t) \sin(\frac{n \pi t}{L}) d t
		\label{eq:Fourier_coeff}
	\end{aligned}
\end{equation}
当 \(L \to 0\),若 \(f(x)\) 绝对可积,则
\begin{equation*}
	a_0 = \lim_{L \to 0}
	\left(	\frac{1}{L} \int_{-L}^{L} f(x) \frac{a_0}{2} \dd x \right)= 0
\end{equation*}
所以
\begin{equation*}
	f(x) =
	\sum_{n=1}
	\left(
	a_n \cos(\frac{n \pi x}{L})  +
	b_n \sin(\frac{n \pi x}{L})
	\right)
\end{equation*}
将 Eq.~\eqref{eq:Fourier_coeff} 代入
\begin{equation*}
	\begin{aligned}
		f(x) & =
		\sum_{n=1}
		\left[
			\left(
			\frac{1}{L} \int_{-L}^{L}  f(t) \cos(\frac{n \pi t}{L}) d t
			\right)
			\cos(\frac{n \pi x}{L})  +
			\left(\frac{1}{L} \int_{-L}^{L}  f(t) \sin(\frac{n \pi t}{L}) d t \right)
			\sin(\frac{n \pi x}{L})
		\right]  \\
		     & =
		\sum_{n=1}
		\frac{1}{L}\left[
			\left(
			\int_{-L}^{L}  f(t) \cos(\frac{n \pi t}{L}) d t
			\right)
			\cos(\frac{n \pi x}{L})  +
			\left( \int_{-L}^{L}  f(t) \sin(\frac{n \pi t}{L}) d t \right)
			\sin(\frac{n \pi x}{L})
			\right]
	\end{aligned}
\end{equation*}
令\( \omega_n = \frac{n \pi}{L}, \Delta \omega = \omega_{n+1} - \omega_n = \frac{\pi}{L}\),当 \(L \to \infty, \Delta \omega \to 0\)
\begin{equation*}
	\sum_0^{+\infty} \Delta \omega \ldots
	=
	\int_0^{+\infty} \dd \omega \ldots
\end{equation*}
\begin{equation*}
	\begin{aligned}
		f(x)
		 & =
		\sum_{n=1}
		\frac{\Delta \omega}{\pi}\left[
			\left(
			\int_{-L}^{L}  f(t) \cos(\frac{n \pi t}{L}) d t
			\right)
			\cos(\frac{n \pi x}{L})  +
			\left( \int_{-L}^{L}  f(t) \sin(\frac{n \pi t}{L}) d t \right)
			\sin(\frac{n \pi x}{L})
		\right]            \\
		 & =
		\int_0^{+\infty}
		\frac{\dd \omega}{\pi}\left[
			\left(
			\int_{-\infty}^{+\infty}  f(t) \cos(\omega t) d t
			\right)
			\sin(\omega x)  +
			\left( \int_{-\infty}^{+\infty}  f(t) \sin(\omega t) d t \right)
			\sin(\omega x)
		\right]            \\
		 & =
		\frac{1}{ \pi}
		\int_{-\infty}^{+\infty} \dd t
		\int_{0}^{+\infty} \dd \omega
		f(t)
		\left(
		\cos(\omega t) \cos(\omega x)
		+
		\sin(\omega t) \sin(\omega x)
		\right)            \\
		 & =
		\frac{1}{ \pi}
		\int_{-\infty}^{+\infty} \dd t
		\int_{0}^{+\infty} \dd \omega
		f(t)
		\cos(\omega (x-t)) \\
		 & =
		\frac{1}{ 2 \pi}
		\int_{-\infty}^{+\infty} \dd t
		\int_{0}^{+\infty} \dd \omega
		f(t)
		\left(
		e^{i \omega (x-t)}
		+
		e^{-i \omega (x-t)}
		\right)            \\
		 & =
		\frac{1}{2 \pi}
		\int_{-\infty}^{+\infty} \dd t
		\int_{-\infty}^{+\infty} \dd \omega
		f(t)
		e^{i \omega (x-t)} \\
		 & =
		\frac{1}{2 \pi}
		\int_{-\infty}^{+\infty} \dd \omega
		\int_{-\infty}^{+\infty} \dd t
		f(t)
		e^{i \omega (x-t)} \\
		 & =
		\frac{1}{2 \pi}
		\int_{-\infty}^{+\infty} \dd \omega
		e^{i \omega x}
		\underbrace{
			\int_{-\infty}^{+\infty} \dd t
			f(t)
			e^{-i \omega t}}_{F(\omega)}
		\\
	\end{aligned}
\end{equation*}
从时域到频域
\begin{equation*}
	F(\omega)=\mathcal{F}[f(t)]
	=
	\int_{-\infty}^{+\infty}
	f(t)
	e^{-i \omega t} \dd t
\end{equation*}
从频域到时域
\begin{equation*}
	f(t)=\mathcal{F}^{-1}[F(\omega)]
	=
	\frac{1}{2 \pi}
	\int_{-\infty}^{+\infty}
	F(\omega)
	e^{i \omega t} \dd \omega
\end{equation*}
函数的定义包括什么
\begin{itemize}
	\item 法则
	\item 自变量
	\item 因变量
	\item 值域
\end{itemize}
\(\mathcal{F}\)是一个算子,把一个函数变为另一个函数。实际上应该有:\(F=\fourier{f}\),不应该写自变量\(F(\omega)=\fourier{f(t)}\)

\(\omega \to \frac{p}{\hbar}\) \\
坐标空间到动量空间
\begin{equation*}
	C(p)
	=
	\frac{1}{\sqrt{2 \pi \hbar}}
	\int_{-\infty}^{+\infty}
	\psi(x)
	e^{- \frac{i}{\hbar} p x}  \dd x
\end{equation*}
动量空间到坐标空间
\begin{equation*}
	\psi(x)
	=
	\frac{1}{\sqrt{2 \pi \hbar}}
	\int_{-\infty}^{+\infty}
	C(p)
	e^{- \frac{i}{\hbar} p x} \dd p
\end{equation*}

\begin{theorem}[狄利克雷定理]
	狄利克雷定理是傅里叶级数收敛的充分条件
	\begin{itemize}
		\item 周期函数
		\item 分段光滑
	\end{itemize}
\end{theorem}

\subsection{卷积}
\begin{equation}
  \begin{aligned}
     \fourier{f}(s) \fourier{g}(s)
    &= \int f(t) e^{- 2 \pi ist} \dd{t}\\
    &=\int g(x) e^{- 2 \pi i sx} \dd{x} \\
    &= \int \int f(t) g(x) e^{- 2 \pi i s(x+t)} \dd{t} \dd{x} \\
    &= \int \int f(u-x) g(x) e^{- 2 \pi i s u} \dd{u} \dd{x} \\
    &= \int \left[\int f(u-x) g(x) \dd{x}\right] e^{- 2 \pi i s u} \dd{u} \\
    &= \int  \left[f*g\right] e^{- 2 \pi i s u} \dd{u} \\
    &= \fourier{f*g}(s)
  \end{aligned}
\end{equation}
\begin{remark}
  \begin{itemize}
    \item   t: time
    \item s: spectrum
    \item x: 坐标
    \item p: 动量
  \end{itemize}
\end{remark}

% TODO
\begin{remark}
为什么变换叫变换
\begin{itemize}
  \item MIT Laplace 变换课程说:自变量变了,如傅立叶变换 \(t\)变成了\(\omega\)
  \item \verb|3b1b| 从可视化角度,变换是可操作的,比如线性变换,对空间的操作
\end{itemize}
\end{remark}

%%% vim: set ts=2 sts=2 sw=2 isk+=\: et cc=+1 formatoptions+=mM:

%! TEX root = ./main.tex
% 齐次性叠加性线性

\begin{definition}[齐次性]
等价于算符和纯数的交换律
\begin{equation}
	\Ham[kf] = k \Ham[f]
\end{equation}
\begin{equation}
	\Ham k f = k \Ham f
\end{equation}
\end{definition}

\begin{definition}[叠加性]
等价于算符的结合律
\begin{equation}
	\Ham[f_1+f_2] = \Ham[f_1] + \Ham[f_2]
\end{equation}
\begin{equation}
	\Ham(f_1+f_2) = \Ham f_1 + \Ham f_2
\end{equation}
\end{definition}


\begin{definition}[线性]
同时满足齐次性和叠加性,即类比线性空间的向量,矩阵作用于向量上,向量数乘和向量加法
\begin{equation}
	\Ham( k_1 f_1 + k_2 f_2 )
	=
	\Ham k_1 f_1 
	+
	\Ham k_2 f_2 
	=
	 k_1\Ham f_1 
	+
	 k_2\Ham f_2 
\end{equation}
\end{definition}


位置(空间)不变性,即退化算符$\Ham$和平移算符$\hat{T}$对易
\begin{definition}[平移算符]
	\begin{equation}
		\T f(x) =  f(x-a)
	\end{equation}
\end{definition}
\begin{equation}
	\Ham \T f(x) = \T \Ham f(x) 
\end{equation}

% debug

% preamble

\documentclass{article}
% \usepackage{ulem}
\usepackage[fontset=windows]{ctex}
\usepackage{NotesCTeX,lipsum}
\usepackage{svg}
\usepackage{outlines}
\usepackage{braket}
\usepackage{siunitx}
\usepackage{wrapfig}
% \usepackage{tabular}
\newcommand{\im}{\mathrm{i}}%
\newcommand{\XX}{\hat{X}}%
\newcommand{\JJ}{\hat{J}}%
\newcommand{\RR}{{\mathbb{R}}}
\newcommand{\RHat}{\hat{R}}%
\newcommand{\tp}{^{\textup{T}}} %矩阵转置 % 覆盖 physics 宏包的 Trace

% \usepackage[english]{babel}
% https://github.com/gpoore/minted/issues/159
\usepackage[outputdir=latex.out]{minted}
\usemintedstyle{mathematica}
\definecolor{LightGray}{gray}{0.95}

% \code{language}{filename}
% eg: \code{wolfram}{./a.mma}
% eg: \code{python}{./a.py}
\newcommand{\code}[2]{%
\inputminted[%
linenos,
baselinestretch=1.2,
bgcolor=LightGray]{#1}{#2}}
\usepackage[capitalise]{cleveref}
\crefname{equation}{式}{式} 
\crefname{figure}{图}{图} 
\crefname{table}{表}{表} 
\crefname{appendix}{附录}{附录} 
% \crefname{enumi}{条目}{条目} 
\newcommand{\crefpairconjunction}{~和~} 
\newcommand{\crefmiddleconjunction}{、} 
\newcommand{\creflastconjunction}{~和~} 
\newcommand{\crefpairgroupconjunction}{~和~} 
\newcommand{\crefmiddlegroupconjunction}{、} 
\newcommand{\creflastgroupconjunction}{~和~} 
\newcommand{\crefrangeconjunction}{~} 

\DeclareMathOperator{\rref}{rref}
% \renewcommand{\bigO}[1]{$\mathcal{O}(#1)$}
\newcommand{\red}[1]{\textcolor{red}{#1}}
\newcommand{\blue}[1]{\textcolor{blue}{#1}}
% highlight
\newcommand{\hl}[1]{\colorbox{BurntOrange}{#1}}
\newcommand{\matr}[1]{\mathbf{#1}} % undergraduate algebra version
% \newcommand{\matr}[1]{\bm{#1}}     % ISO complying version

% https://tex.stackexchange.com/questions/227639/redefine-emph-to-be-both-bold-and-italic
\let\emph\relax % there's no \RedeclareTextFontCommand
\DeclareTextFontCommand{\emph}{\color{red}\em}

% preamble

\documentclass{article}
% \usepackage{ulem}
\usepackage[fontset=windows]{ctex}
\usepackage{NotesCTeX,lipsum}
\usepackage{svg}
\usepackage{outlines}
\usepackage{braket}
\usepackage{siunitx}
\usepackage{wrapfig}
% \usepackage{tabular}
\newcommand{\im}{\mathrm{i}}%
\newcommand{\XX}{\hat{X}}%
\newcommand{\JJ}{\hat{J}}%
\newcommand{\RR}{{\mathbb{R}}}
\newcommand{\RHat}{\hat{R}}%
\newcommand{\tp}{^{\textup{T}}} %矩阵转置 % 覆盖 physics 宏包的 Trace

% \usepackage[english]{babel}
% https://github.com/gpoore/minted/issues/159
\usepackage[outputdir=latex.out]{minted}
\usemintedstyle{mathematica}
\definecolor{LightGray}{gray}{0.95}

% \code{language}{filename}
% eg: \code{wolfram}{./a.mma}
% eg: \code{python}{./a.py}
\newcommand{\code}[2]{%
\inputminted[%
linenos,
baselinestretch=1.2,
bgcolor=LightGray]{#1}{#2}}
\usepackage[capitalise]{cleveref}
\crefname{equation}{式}{式} 
\crefname{figure}{图}{图} 
\crefname{table}{表}{表} 
\crefname{appendix}{附录}{附录} 
% \crefname{enumi}{条目}{条目} 
\newcommand{\crefpairconjunction}{~和~} 
\newcommand{\crefmiddleconjunction}{、} 
\newcommand{\creflastconjunction}{~和~} 
\newcommand{\crefpairgroupconjunction}{~和~} 
\newcommand{\crefmiddlegroupconjunction}{、} 
\newcommand{\creflastgroupconjunction}{~和~} 
\newcommand{\crefrangeconjunction}{~} 

\DeclareMathOperator{\rref}{rref}
% \renewcommand{\bigO}[1]{$\mathcal{O}(#1)$}
\newcommand{\red}[1]{\textcolor{red}{#1}}
\newcommand{\blue}[1]{\textcolor{blue}{#1}}
% highlight
\newcommand{\hl}[1]{\colorbox{BurntOrange}{#1}}
\newcommand{\matr}[1]{\mathbf{#1}} % undergraduate algebra version
% \newcommand{\matr}[1]{\bm{#1}}     % ISO complying version

% https://tex.stackexchange.com/questions/227639/redefine-emph-to-be-both-bold-and-italic
\let\emph\relax % there's no \RedeclareTextFontCommand
\DeclareTextFontCommand{\emph}{\color{red}\em}

% preamble

\documentclass{article}
% \usepackage{ulem}
\usepackage[fontset=windows]{ctex}
\usepackage{NotesCTeX,lipsum}
\usepackage{svg}
\usepackage{outlines}
\usepackage{braket}
\usepackage{siunitx}
\usepackage{wrapfig}
% \usepackage{tabular}
\newcommand{\im}{\mathrm{i}}%
\newcommand{\XX}{\hat{X}}%
\newcommand{\JJ}{\hat{J}}%
\newcommand{\RR}{{\mathbb{R}}}
\newcommand{\RHat}{\hat{R}}%
\newcommand{\tp}{^{\textup{T}}} %矩阵转置 % 覆盖 physics 宏包的 Trace

% \usepackage[english]{babel}
% https://github.com/gpoore/minted/issues/159
\usepackage[outputdir=latex.out]{minted}
\usemintedstyle{mathematica}
\definecolor{LightGray}{gray}{0.95}

% \code{language}{filename}
% eg: \code{wolfram}{./a.mma}
% eg: \code{python}{./a.py}
\newcommand{\code}[2]{%
\inputminted[%
linenos,
baselinestretch=1.2,
bgcolor=LightGray]{#1}{#2}}
\usepackage[capitalise]{cleveref}
\crefname{equation}{式}{式} 
\crefname{figure}{图}{图} 
\crefname{table}{表}{表} 
\crefname{appendix}{附录}{附录} 
% \crefname{enumi}{条目}{条目} 
\newcommand{\crefpairconjunction}{~和~} 
\newcommand{\crefmiddleconjunction}{、} 
\newcommand{\creflastconjunction}{~和~} 
\newcommand{\crefpairgroupconjunction}{~和~} 
\newcommand{\crefmiddlegroupconjunction}{、} 
\newcommand{\creflastgroupconjunction}{~和~} 
\newcommand{\crefrangeconjunction}{~} 

\DeclareMathOperator{\rref}{rref}
% \renewcommand{\bigO}[1]{$\mathcal{O}(#1)$}
\newcommand{\red}[1]{\textcolor{red}{#1}}
\newcommand{\blue}[1]{\textcolor{blue}{#1}}
% highlight
\newcommand{\hl}[1]{\colorbox{BurntOrange}{#1}}
\newcommand{\matr}[1]{\mathbf{#1}} % undergraduate algebra version
% \newcommand{\matr}[1]{\bm{#1}}     % ISO complying version

% https://tex.stackexchange.com/questions/227639/redefine-emph-to-be-both-bold-and-italic
\let\emph\relax % there's no \RedeclareTextFontCommand
\DeclareTextFontCommand{\emph}{\color{red}\em}

\input{../preamble}
\input{choice_question_env}


%%% vim: set ts=2 sts=2 sw=2 isk+=\: et cc=+1 formatoptions+=mM:

% choice_question_env
\newlength{\xxla}
\newlength{\xxlb}
\newlength{\xxlc}
\newlength{\xxld}
\newlength{\lhalf}
\newlength{\lquarter}
\newlength{\lmax}
\newcommand{\options}[4]{
	\par
	\settowidth{\xxla}{(A)~#1~~~}
	\settowidth{\xxlb}{(B)~#2~~~}
	\settowidth{\xxlc}{(C)~#3~~~}
	\settowidth{\xxld}{(D)~#4~~~}

	\ifthenelse{\lengthtest{\xxla>\xxlb}}{\setlength{\lmax}{\xxla}}{\setlength{\lmax}{\xxlb}}
	\ifthenelse{\lengthtest{\xxlc>\lmax}}{\setlength{\lmax}{\xxlc}}{}
	\ifthenelse{\lengthtest{\xxld>\lmax}}{\setlength{\lmax}{\xxld}}{}
	\setlength{\lhalf}{0.5\linewidth}
	\setlength{\lquarter}{0.25\linewidth}
	\ifthenelse{\lengthtest{\lmax>\lhalf}}
	{
		\begin{enumerate}[label=(\Alph*),parsep=0ex,itemsep=0ex,leftmargin=2em, topsep=0ex]
			\item #1
			\item #2
			\item #3
			\item #4
		\end{enumerate}
	}
	{
		\ifthenelse{\lengthtest{\lmax>\lquarter}}
		{
			\makebox[\lhalf][l]{(A)~#1~~~}%
			\makebox[\lhalf][l]{(B)~#2~~~}%

			\makebox[\lhalf][l]{(C)~#3~~~}%
			\makebox[\lhalf][l]{(D)~#4~~~}%
		}
		{
			\makebox[\lquarter][l]{(A)~#1~~~}%
			\makebox[\lquarter][l]{(B)~#2~~~}%
			\makebox[\lquarter][l]{(C)~#3~~~}%
			\makebox[\lquarter][l]{(D)~#4~~~}%
		}
	}
}




%%% vim: set ts=2 sts=2 sw=2 isk+=\: et cc=+1 formatoptions+=mM:



%%% vim: set ts=2 sts=2 sw=2 isk+=\: et cc=+1 formatoptions+=mM:

% choice_question_env
\newlength{\xxla}
\newlength{\xxlb}
\newlength{\xxlc}
\newlength{\xxld}
\newlength{\lhalf}
\newlength{\lquarter}
\newlength{\lmax}
\newcommand{\options}[4]{
	\par
	\settowidth{\xxla}{(A)~#1~~~}
	\settowidth{\xxlb}{(B)~#2~~~}
	\settowidth{\xxlc}{(C)~#3~~~}
	\settowidth{\xxld}{(D)~#4~~~}

	\ifthenelse{\lengthtest{\xxla>\xxlb}}{\setlength{\lmax}{\xxla}}{\setlength{\lmax}{\xxlb}}
	\ifthenelse{\lengthtest{\xxlc>\lmax}}{\setlength{\lmax}{\xxlc}}{}
	\ifthenelse{\lengthtest{\xxld>\lmax}}{\setlength{\lmax}{\xxld}}{}
	\setlength{\lhalf}{0.5\linewidth}
	\setlength{\lquarter}{0.25\linewidth}
	\ifthenelse{\lengthtest{\lmax>\lhalf}}
	{
		\begin{enumerate}[label=(\Alph*),parsep=0ex,itemsep=0ex,leftmargin=2em, topsep=0ex]
			\item #1
			\item #2
			\item #3
			\item #4
		\end{enumerate}
	}
	{
		\ifthenelse{\lengthtest{\lmax>\lquarter}}
		{
			\makebox[\lhalf][l]{(A)~#1~~~}%
			\makebox[\lhalf][l]{(B)~#2~~~}%

			\makebox[\lhalf][l]{(C)~#3~~~}%
			\makebox[\lhalf][l]{(D)~#4~~~}%
		}
		{
			\makebox[\lquarter][l]{(A)~#1~~~}%
			\makebox[\lquarter][l]{(B)~#2~~~}%
			\makebox[\lquarter][l]{(C)~#3~~~}%
			\makebox[\lquarter][l]{(D)~#4~~~}%
		}
	}
}




%%% vim: set ts=2 sts=2 sw=2 isk+=\: et cc=+1 formatoptions+=mM:



%%% vim: set ts=2 sts=2 sw=2 isk+=\: et cc=+1 formatoptions+=mM:

\title{等离子体物理基础第三次作业}
\date{\today}
\author{核院\ 徐均益\\ SA22214015}
\begin{document}
任意磁场可以由两个标量场\(\alpha,\beta\)表示,\(\alpha,\beta\)和空间、时间都有关\sn{散度为零保证了}
\begin{equation}
  \vb{B} = \grad \alpha \cross \grad \beta
\end{equation}
磁场可以看作一根根磁力线,磁力线可以看作两个曲面的交线,在数学上如何表述一条曲线? % TODO

冻结方程可以写作
\begin{equation}
  \pdv{t}( \grad \alpha \cross \grad \beta ) = \curl( u \cross ( \grad \alpha \cross \grad \beta ) )
\end{equation}
也就是
\begin{equation}
  \grad(\pdv{\alpha}{t} \cross \grad \beta) + \grad \alpha + \grad(\pdv{\beta}{t}) = 
  \curl( - \vb{u} \vdot \grad \alpha \grad \beta + \vb{u} \vdot \grad \beta \grad \alpha  )
\end{equation}

\section{流体漂移}

\section{应力张量}
\(T_{ij}\),其中\(i\)表示方向, \(j\)表示作用于哪个面,是\(\Delta y \Delta z\)(法向沿着\(x\)轴)还是\(\Delta z \Delta x\)还是\(\Delta x \Delta y\);\(\vu{n}\)表示选取的面元的法向,\(T_{ij}n_j\)把\(j\)缩并后,表示的是穿过面元的力的\(i\)分量。
如果不存在切向力,那么\(T_{ij}\)是一个对角矩阵。

The force \(\vb{F}_1\) across an element of area \(\Delta y \Delta z\) perpendicular to \(x\)-axis 
is resolved into three components  \( F_{x1}, F_{y1}, F_{z1}\)

\section{相速度乘群速度等于光速平方}
\begin{tabular}{ll}
  相速度 & phase velocity  \\
  群速度 & group velocity
\end{tabular}
\begin{equation}
  v_p v_g = c^2
\end{equation}
\begin{proof}
\begin{equation}
  v_p = 
  \frac{\omega}{k} 
  =
  \frac{\hbar \omega}{\hbar k} 
  = \frac{E}{p}
\end{equation}
\begin{equation}
  \begin{aligned}
    v_g &= 
  \pdv{\omega}{k} = 
  \pdv{\hbar \omega}{\hbar k} \\ 
        &= \pdv{E}{p} = \frac{p}{m} \qq{利用} E^2 = (pc)^2 + (m_0 c^2)^2
  \end{aligned}
\end{equation}
\end{proof}

加热,放电,大电流,电阻,
等离子体特性,温度越高,电阻越低,几百万度,电阻已经很小了,
微波加热,中性束加热


研究均匀等离子体,密度温度磁场都均匀 

\begin{equation*}
  \div \vb{B} = 0
\end{equation*}
\begin{equation}
  B_r = -B_0 \alpha^2 z r
\end{equation}

% 勒让德多项式

\section{勒让德多项式}%
\begin{equation}
  \laplacian \varphi = 0
\end{equation}
由 \cref{eq:广义曲线坐标系拉普拉斯算子} 写出求坐标系的拉普拉斯算子,
\begin{equation}
  \laplacian = 
  \frac{1}{r^2} \pdv{r} r^2 \pdv{r} +
  \frac{1}{r^2} \pdv{\theta} \sin \theta \pdv{\theta} +
  \frac{1}{r^2 \sin \theta} \pdv[2]{\phi}
  \label{eq:球坐标系的拉普拉斯算子}
\end{equation}
当函数$\varphi(r, \theta)$与\(\phi\)无关的时候
\begin{equation}
  \begin{aligned}
    \laplacian &= 
  \frac{1}{r^2} \pdv{r} r^2 \pdv{r} +
  \frac{1}{r^2 \sin\theta} \pdv{\theta} \sin \theta \pdv{\theta} +
  \frac{1}{r^2 \sin \theta} \pdv[2]{\phi} \\
\text{(当函数$\varphi(r, \theta)$与\(\phi\)无关的时候)}
\quad &=
  \frac{1}{r^2} \pdv{r} r^2 \pdv{r} +
  \frac{1}{r^2 \sin\theta} \pdv{\theta} \sin \theta \pdv{\theta}
  \end{aligned}
\end{equation}
\begin{equation}
  \begin{aligned}
   & \left(\cancel{\frac{1}{r^2}} \pdv{r} r^2 \pdv{r} +
  \frac{1}{\cancel{r^2} \sin\theta} \pdv{\theta} \sin \theta \pdv{\theta}\right)
  \varphi \\
  &=
  \left(\pdv{r} r^2 \pdv{r} +
  \frac{1}{\sin\theta} \pdv{\theta} \sin \theta \pdv{\theta}\right)
  \varphi = 0
  \end{aligned}
\end{equation}
若 \(\varphi(r, \theta) = R(r) \Theta(\theta)\)
\begin{equation}
 \frac{1}{R} \red{\left(\pdv{r} r^2 \pdv{r} \right) }R=
-\frac{1}{\Theta} \blue{\left(\frac{1}{\sin\theta} \pdv{\theta} \sin \theta \pdv{\theta}\right)} \Theta
  = \lambda
\end{equation}
\begin{align}
  \red{\left(\dv{r} r^2 \dv{r} \right) } R = \lambda R \label{eq:Legendre_R}\\
\blue{\left(\frac{1}{\sin\theta} \dv{\theta} \sin \theta \dv{\theta}\right)}\Theta =  - \lambda \Theta \label{eq:Legendre_Q}
\end{align}
\cref{eq:Legendre_R} 即
\begin{equation}
  \red{\left(\dv{r} r^2 \dv{r} \right) } R = \lambda R 
\end{equation}
的 $r> 0$ 而指数函数\(e^t>0, t\in \RR\),不妨设\(r=e^t\)。
由于
链式法则
\begin{equation}
  \begin{aligned}
  &\dv{f}{r}= \dv{t}{r} \dv{f}{t} = \frac{1}{r} \dv{f}{t} \\
  \implies &\dv{r}=\frac{1}{r} \dv{t}
  \end{aligned}
\end{equation}
得
\begin{equation}
  \red{\left(\dv{r} r^2 \dv{r} \right) } R=
\frac{1}{r} \dv{t} r^2 \frac{1}{r} \dv{t} R
=
\frac{1}{r} \dv{t} r\dv{t} R
= \frac{1}{r} \left( r \ddot{R} + r \dot{R} \right) = \lambda R
\end{equation}
\begin{equation}
\ddot{R} + \dot{R} - \lambda R = 0
\end{equation}
目前\footnote{后面可知,只有当\(l\)为整数时,才能截断}\(\lambda \in \RR\),不妨设\(\lambda = l(l+1), l \in \RR\),
解得
\begin{subequations}
  \begin{align}
  &R_1 = C_1 e^{lt} = C_1 r^l \\
  &R_2 = C_2 e^{(-l-1)t} = C_2 r^{-l-1}
\end{align}
\end{subequations}
\begin{equation*}
	\begin{aligned}
		P_l(x) &= \frac{1}{2^l} \sum_{m=0}^{[\frac{l}{2}]}
	(-1)^m C_l^m C_{2l - 2m}^{l} x^{l - 2m}\\
&= \frac{1}{2^l} \sum_{m=0}^{[\frac{l}{2}]}
	(-1)^m \frac{(2l - 2m)!}{m! (l-m)! (l-2m)!} x^{l-2m}
	\end{aligned}
\end{equation*}
其中\([\ ]\)代表向下取整
\begin{equation*}
	\left[\frac{l}{2}\right] = 
	\begin{cases}
		\frac{l}{2} & \qq{\(l\) is even } \\
		\frac{l-1}{2} & \qq{\(l\) is odd }
	\end{cases}
\end{equation*}

\begin{align*}
	P_0 = 1 \\
	P_1 = x \\
	P_2 = \frac{1}{2}(3x^2 - 1) \\
	P_3 = \frac{1}{2}(5x^3 - 3x)
\end{align*}





%%% vim: set ts=2 sts=2 sw=2 isk+=\: et cc=+1 formatoptions+=mM:



\end{document}

%%% vim: set ts=2 sts=2 sw=2 isk+=\: et cc=+1 formatoptions+=mM:

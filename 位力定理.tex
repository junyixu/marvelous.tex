% 位力定理

\section{位力定理}

位力定理\footnote{见朗道力学第二章第10节}:
若势能\(U \propto r^k\),则有
\begin{equation}
  \Braket{kU} \equiv \Braket{\vb{r} \vdot \pdv{U}{\vb{r}}} =
  \Braket{ \vb{v} \vdot \pdv{T}{\vb{v}}} \equiv  \Braket{2T}
\end{equation}
中间的等号是因为牛二。多粒子系统要写\(\Sigma_\alpha\)。
\begin{proof} (由牛二证明)
  \begin{equation}
    \vb{F} = m \vb{a}
  \end{equation}
  两边同时乘以 \(\vb{r}\)
  \begin{equation}
    \begin{aligned}
      \vb{r} \vdot \left( - \pdv{U}{\vb{r}} \right) &= m \dv{\vb{v}}{t} \vdot \vb{r} \\
                                                    &=
m \dv{t}(\vb{v} \vdot \vb{r})
- m \vb{v} \vdot \vb{v}
    \end{aligned}
  \end{equation}
    两边同时在一个周期内对时间做平均,由于$ m \dv{t}(\vb{v} \vdot \vb{r})$ 是一个全微分,初态和末态在相空间的值$(\vb{r}, \vb{v})$是一样的,故只剩下
  \begin{equation}
    \Braket{\vb{r} \vdot \pdv{U}{\vb{r}} } =
\Braket{m \vb{v} \vdot \vb{v}}
  \end{equation}
\end{proof}




%%% vim: set ts=2 sts=2 sw=2 isk+=\: et cc=+1 formatoptions+=mM:
